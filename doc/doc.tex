\documentclass{article}
\usepackage{graphicx}
\usepackage{polski}
\usepackage[cp1250]{inputenc}
\usepackage{url}
\usepackage[margin=2.5cm]{geometry} %layout
% the following is needed for syntax highlighting
\usepackage{color}
\usepackage{listings} % needed for the inclusion of source code
\usepackage{amsmath}


\definecolor{dkgreen}{rgb}{0,0.6,0}
\definecolor{gray}{rgb}{0.5,0.5,0.5}
\definecolor{mauve}{rgb}{0.58,0,0.82}


\begin{document}

\begin{titlepage}
\begin{center}

\textsc{\LARGE Programowanie urządzeń mobilnych pod kątem zastosowań biometrycznych}\\[1.5cm]

\vskip 3cm

\textsc{\Large Sprawozdanie I}\\[0.5cm]

{ \huge \bfseries Temat: EKG \\[0.4cm] }

\vskip 6cm
% Author and supervisor
\begin{minipage}{0.4\textwidth}
\begin{flushleft} \large
\emph{Autor:}\\
Paweł Pęksa
\end{flushleft}
\end{minipage}
\begin{minipage}{0.4\textwidth}
\begin{flushright} \large
\emph{Prowadzący ćwiczenia:} \\
mgr Krzysztof Misztal
\end{flushright}
\end{minipage}

\vfill

% Bottom of the page
{\large Kraków, \today
}

\end{center}
\end{titlepage}

\tableofcontents
\clearpage

\section{Temat zadania}
Celem projektu było wyświetlenie fragmentu EKG na urządzeniu z systemem Android.

 
\section{Proponowane rozwiązanie}
Aplikacja została napisana z użyciem biblioteki do rysowania wykresów AndroidPlot. Dane stanowiły dwie próbki pobrany z bazy dostępnej w internecie, a konkretniej ich pierwsze 4 sekundy.

\subsection{Krok 1: Wczytanie danych z pliku}
Dane były wczytywane z pliku z pomocą klasy BufferedReader dostępnej w bibliotece javy. Pierwsze dwie linie pliku CSV stanowiące nagłówek były pomijane, następnie przetwarzane były kolejne linijki tak, aby móc swobodnie je umieścić w obiektach klasy ArrayList stanowiących odpowiednio kontenery danych dla dziedziny oraz wartości funkcji.

\subsubsection{Czynności}
Kod metody realizującej wczytywanie danych z pliku


\subsection{Krok 2: Rysowanie wykresu}
Rysowanie wykresu realizowane z pomocą biblioteki AndroidPlot polegało na stworzeniu obiketów dla każdej funkcji, które chcieliśmy przedstawić oraz stworzeniu dla nich odpowiednich formatów w celu wyświetlania.

\subsubsection{Czynności}
Kod metody realizującej rysowanie wykresu


\subsubsection{Rezultat}
Tu jest zakomentowany obrazek, jakby trzeba było to tak można właśnie go wrzucić
%\begin{center}
%    \centering
%    \includegraphics[scale=0.5]{res/ekg.png}
%    \\Wykresy EKG
%    \label{simulationfigure}
%\end{center}


\section{Wnioski}
Podumowując, aplikacja spełnia swoje zadanie. Wykresy są rysowane poprawnie. 
\clearpage
\addcontentsline{toc}{section}{Literatura}
\begin{thebibliography}{9}

\bibitem{Simpson} Strona laboratorium 
\url{misztal.edu.pl}

\bibitem{Simpson} Biblioteka AndroidPlot
\url{androidplot.com}


\bibitem{Simpson} Baza danych EKG
\url{www.physionet.org/cgi-bin/atm/ATM}

\end{thebibliography}

\end{document}
